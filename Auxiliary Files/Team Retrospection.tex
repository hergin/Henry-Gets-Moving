\documentclass[12pt, letterpaper]{article}
\renewcommand{\familydefault}{\sfdefault}
\setcounter{secnumdepth}{0}
\usepackage{graphicx}
\title{team retrospection (epic)}
\date{eventually}
\author{melody, mars, and kiri}

\begin{document}
\maketitle
\section{team photo}
\includegraphic{teamphoto}

\section{requirement match}
\begin{table}
\center
\begin{tabular}{|p{5.5in}|p{1in}|}\hline
	\textbf{initial features} & \textbf{implemented?}&\hline
	Users should be able to log children's exercise, including time exercised and type of exercise & Included&\hline
	Should include a page for healthy recipes, with low-cost, simple recipes & Included &\hline
	Includes a daily Recipe of the Day which changes every day &Included&\hline
	Recipes should be filterable by ingredients they include& Included (categories, search)&\hline
	A maze game, similar to the ones the immersive learning project already has on paper&Meh (included the paper)&\hline
	Demonstrations of healthy alternatives to unhealthy choices, and information as to why said choices are unhealthy & Included&\hline
	An `About' page, including information about the immersive learning project from the Henry Gets Moving Delaware County Facebook page, with a picture of the immersive learning team&Included&\hline
	A list of exercises and information about them grouped into sections based on the type&Included&\hline
	An Exercise of the Day which changes every day&Included&\hline
	Users should have an account linked to a username which stores their exercise log data&Included&\hline
	Information about which fruits and vegetables are currently in season to go with the healthy recipes&Not included&\hline
	An administrator section where daily recipes and exercises can be interchanged and the catalogs for each can be updated&Included&\hline
	The app should be accessible to both children and guardians to use&Yep&\hline
	Design elements from the book and the immersive learning project should be included&Yep&\hline
	Include pictures of the characters from the book&Yep&\hline
	Provide a small celebration when 60 minutes of exercise is logged&Yep (trophy)&\hline
	The app should be functional both on desktop and on mobile, as it will primarily be used on iPads&Yep&\hline
	Language written at a pre-school ~ second grade level to be easy for kids to read&Yep&\hline
	Site should be colorful and designed in a way appealing to kids&Yep&\hline

\end{tabular}\caption{requirement match table}
\end{table}
\subsection{If there are missing features in the software, why?}
	The two we missed out on (the maze, fruits/veggies in season) were both marked as low-priority on our requirements list, and the time constraints suggested we would have been better off adding other things and avoiding stepping outside of our knowledge bases. We had a lot of responsibilities on top of adding new features, and we saw other things that would have been useful to our client, that she confirmed were also useful (and were easier to implement).

\section{final meeting and transfer}
\subsection{how did it go?}

\subsection{\emph{transfer}: how did you do it?}

\section{retrospection}
\subsection{What actions/practices/steps did you take successfully and do them again in your next software project?}

\subsection{What actions/practices/steps did you take but you are not happy with the result and wouldn't do them again in your next software project?}

\subsection{What actions/practices/steps did you miss taking but you would do in your next software project?}

\subsection{Suggestions for the future Capstone Teams.}

\subsection{Final thoughts on the project.}
\subsubsection{Each team member will write about how they feel about the status of the project and the two-semester-long process.}
\subsubsection{It can be related to anything about the project.}
\begin{itemize}
	\item{\textbf{Melody}}
	
		For me, personally, it could've gone better. Of course, I learned React, which should be a valuable skill, 
		as all I knew before this class was bare-bones HTML web development from CS 410, but I could have learned 
		React in considerably less painful ways that didn't involve being graded. I liked the \emph{idea} of a 
		two-semester long class that involved working with an external client, but it would have been much better 
		done if it was actually a culmination of everything we'd learned in Ball State thus far instead of just 
		a totally out-of-nowhere project where we're learning entirely new things we've never seen before, while 
		also simultaneously having to implement the things we've been learning since CS 222 that we'll be graded on but that 
		don't necessarily have anything to do with the project. I think we created a great end product, despite the 
		constraints we were placed under, but I don't imagine this'll be the first thing I cite on my résumé, as it 
		wasn't for other alumni I spoke to who took this class that didn't even put it on their résumé before getting 
		hired.
	
	\item{\textbf{Mars}}
	
		insert text here
	
	\item{\textbf{Kiri}}
	
		insert text here
\end{itemize}
\end{document}
